% Chapter 6

\chapter{Conclusiones y trabajos futuros} % Main chapter title

\label{Capítulo 6} % For referencing the chapter elsewhere, use \ref{Chapter6} 

%----------------------------------------------------------------------------------------

% Define some commands to keep the formatting separated from the content 
%\newcommand{\keyword}[1]{\textbf{#1}}
%\newcommand{\tabhead}[1]{\textbf{#1}}
%\newcommand{\code}[1]{\texttt{#1}}
%\newcommand{\file}[1]{\texttt{\bfseries#1}}
%\newcommand{\option}[1]{\texttt{\itshape#1}}

%----------------------------------------------------------------------------------------





%----------------------------------------------------------------------------------------

Este trabajo final de Master presentado, propone una nueva alternativa para el reconocimiento de iris en condiciones no ideales. La utilización de esta nueva propuesta aporta varias contribuciones que se traducen principalmente en las siguientes direcciones: aplicar un adecuado tratamiento a las imágenes que resalten más el área del iris para realizar una mejor segmentación, un método basado en análisis de perfiles para la segmentación del iris en base a su centro y sus áreas interna y externa, una aplicación al iris segmentado que ayude a realzar los puntos de interés de dicha región y un nuevo método para la extracción de características del iris basado en un esquema de fusión de 3 fuentes de información. De manera general, todos los objetivos fueron alcanzados y se pudieron comprobar sus resultados. A continuación se presentan con más detalles:

\begin{itemize}
    \item Debido a la diversidad de cada una de las imágenes por el tipo de afectación que padece, se hace muy útil aplicar un tratamiento sobre las mismas que ayude a resaltar el área completa del iris frente al resto de la imagen. Para ello se propuso aplicar un filtro Gaussiano adaptando el umbral del mismo a las condiciones del tipo de imagen en función de su afección. De esta manera se consiguió resaltar los detalles del área del iris y eliminar las afecciones producidas en la imagen por las reflexiones especulares, oclusiones e iluminación variable, lo que consiguió hacer mas robusto el proceso de segmentación del mismo..\\
    
    \item Con el área del iris mas resaltada se pasó a realizar la segmentación de la misma basándose en que el centro del iris se acerca al centro de la propia imagen, por lo que a través de un análisis de perfiles localizamos el centro de dicha área. Con el centro del iris localizado, lo siguiente a realizar era la segmentación de los bordes interno y externo de este. Para dicha operación se utilizó un método que buscaba border circulares sobre un radio mínimo y un radio máximo previamente definidos.\\
    
    \item Se realizón un nuevo tratamiento esta vez a la textura del iris segmentado con el objetivo de mejorarla. El método aplicado en cuestión trata de una mejora de contraste para obtener un histograma ecualizado que ayude a mejorar el contraste de las imágenes de iris. Se hicieron varias pruebas para conseguir ajustar adecuadamente el límite del aumento del contraste y evitar una saturación del mismo. \\
    
    \item Se presentó un nuevo método para la extración de características basado en un esquema de fusión de 3 fuentes de información. Este método es muy apropiado para trabajar con imágenes que tienen una calidad variable producida por las diferentes afectaciones. El esquema de fusión propuesto se basa en las ponderaciones obtenidas  del ranking de medidas de desempeño por las 3 fuentes de informaciones. Los resultados que se obtuvieron indicaron la superioridad del nuevo método propuesto frente al resto de métodos del estado del arte. \\
\end{itemize} 


\section{Trabajos futuros}
Son varias las líneas de investigación aparecidas durante el desarrollo del presente trabajo final de Master entre las que se consideran las siguientes:

\begin{itemize}
	 \item Desarrollar una nueva variante del proceso de segmentación del iris haciéndolo más preciso para que sea capaz de eliminar cualquier tipo de afeccón que no se haya podido eliminar en esta investigación. \\
	 
	 \item Ampliar el tipo de imágenes y sus afecciones para ver el comportamiento del método propuesto frente a diferentes factores de calidad.\\
	 
	  \item Desarrollar una variante para el tratamiento de la textura del iris que ayude a resalta con mayor alcance los puntos de interés del mismo.  \\
	  
	  \item Realizar más pruebas de cómputo en una máquina con características superiores a la empleada en esta investigación que permita reducir los tiempo de cálculo. \\
	  
	  \item Desarrollar una adaptación del método propuesto para que la ejecución del mismo y del resto de algoritmos se realicen en GPU en lugar de CPU cmo hasta ahora para que los tiempo se vean considerablemente reducidos.\\
\end{itemize} 

