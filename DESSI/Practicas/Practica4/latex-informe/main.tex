\documentclass[12pt,letterpaper]{article}
\usepackage[utf8]{inputenc}
\usepackage[spanish, es-tabla]{babel}
\usepackage[version=3]{mhchem}
\usepackage[journal=jacs]{chemstyle}
\usepackage{amsmath}
\usepackage{amsfonts}
\usepackage{amssymb}
\usepackage{makeidx}
\usepackage{xcolor}
\usepackage[stable]{footmisc}
\usepackage[section]{placeins}
%Paquetes necesarios para tablas
\usepackage{longtable}
\usepackage{array}
\usepackage{xtab}
\usepackage{multirow}
\usepackage{colortab}
%Paquete para el manejo de las unidades
\usepackage{siunitx}
\sisetup{mode=text, output-decimal-marker = {,}, per-mode = symbol, qualifier-mode = phrase, qualifier-phrase = { de }, list-units = brackets, range-units = brackets, range-phrase = --}
\DeclareSIUnit[number-unit-product = \;] \atmosphere{atm}
\DeclareSIUnit[number-unit-product = \;] \pound{lb}
\DeclareSIUnit[number-unit-product = \;] \inch{"}
\DeclareSIUnit[number-unit-product = \;] \foot{ft}
\DeclareSIUnit[number-unit-product = \;] \yard{yd}
\DeclareSIUnit[number-unit-product = \;] \mile{mi}
\DeclareSIUnit[number-unit-product = \;] \pint{pt}
\DeclareSIUnit[number-unit-product = \;] \quart{qt}
\DeclareSIUnit[number-unit-product = \;] \flounce{fl-oz}
\DeclareSIUnit[number-unit-product = \;] \ounce{oz}
\DeclareSIUnit[number-unit-product = \;] \degreeFahrenheit{\SIUnitSymbolDegree F}
\DeclareSIUnit[number-unit-product = \;] \degreeRankine{\SIUnitSymbolDegree R}
\DeclareSIUnit[number-unit-product = \;] \usgallon{galón}
\DeclareSIUnit[number-unit-product = \;] \uma{uma}
\DeclareSIUnit[number-unit-product = \;] \ppm{ppm}
\DeclareSIUnit[number-unit-product = \;] \eqg{eq-g}
\DeclareSIUnit[number-unit-product = \;] \normal{\eqg\per\liter\of{solución}}
\DeclareSIUnit[number-unit-product = \;] \molal{\mole\per\kilo\gram\of{solvente}}
\usepackage{cancel}
%Paquetes necesarios para imágenes, pies de página, etc.
\usepackage{graphicx}
\usepackage{lmodern}
\usepackage{fancyhdr}
\usepackage[left=4cm,right=2cm,top=3cm,bottom=3cm]{geometry}

%Instrucción para evitar la indentación
%\setlength\parindent{0pt}
%Paquete para incluir la bibliografía
\usepackage[backend=bibtex,style=chem-acs,biblabel=dot]{biblatex}
\addbibresource{references.bib}

%Formato del título de las secciones

\usepackage{titlesec}
\usepackage{enumitem}
\titleformat*{\section}{\bfseries\large}
\titleformat*{\subsection}{\bfseries\normalsize}

%Creación del ambiente anexos
\usepackage{float}
\floatstyle{plaintop}
\newfloat{anexo}{thp}{anx}
\floatname{anexo}{Anexo}
\restylefloat{anexo}
\restylefloat{figure}

%Modificación del formato de los captions
\usepackage[margin=10pt,labelfont=bf]{caption}

%Paquete para incluir comentarios
\usepackage{todonotes}

%Paquete para incluir hipervínculos
\usepackage[colorlinks=true, 
            linkcolor = blue,
            urlcolor  = blue,
            citecolor = black,
            anchorcolor = blue]{hyperref}

%%%%%%%%%%%%%%%%%%%%%%
%Inicio del documento%
%%%%%%%%%%%%%%%%%%%%%%

\begin{document}
\renewcommand{\labelitemi}{$\checkmark$}

\renewcommand{\CancelColor}{\color{red}}

\newcolumntype{L}[1]{>{\raggedright\let\newline\\\arraybackslash}m{#1}}

\newcolumntype{C}[1]{>{\centering\let\newline\\\arraybackslash}m{#1}}

\newcolumntype{R}[1]{>{\raggedleft\let\newline\\\arraybackslash}m{#1}}

\begin{center}
	\textbf{\LARGE{Informe de evaluación}}\\
	\vspace{7mm}
		\textbf{\large{Manuel Jesús García Manday}}\\
	\textbf{\large{ María Victoria Santiago Alcalá}}\\
	\textbf{\large{ Pablo Martín-Moreno Ruiz}}\\
	\textbf{\large{ Mario Ortega Aguayo}}\\
	\vspace{4mm}
	\textbf{\large{Proyecto: GoPark}}\\
	\today
\end{center}

\vspace{7mm}

\section{Resumen}

Este documento refleja el informe de evaluación obtenido en base a los resultados arrojados por los cuestionarios realizados por los diferentes usuarios. En él se puede observar la influencia de cada aspecto de cara al usuario, así como las relaciones entre las mismas y las furtuas mejoras.

\section{Evaluación}

Una vez realizada la evaluación a los diferentes usuarios y conocidos los resultados de las cuestiones presentadas a los mismos sobre los cinco principales aspectos de la aplicación (diseño gráfico, ayuda, facilidad de uso, aprendizaje y satisfacción), vamos a pasar a valorar cada uno de los resultados y proponer las mejoras correspondientes en el aspecto que lo necesite.\\

Mirando la tabla de los valores obtenidos sobre las cuestiones de la aplicación podemos ver como la \textbf{ayuda} tiene una media mas baja que el resto de los aspectos, incluso por debajo de la media del rango en el que oscilan los valores (0-10), lo cual nos quiere decir que este tipo de funcionalidad no realiza bien su cometido por lo que es necesario modificarla para mejorar la usabilidad de cara al usuario final y que pueda entonces ser útil para sus tareas.\\

Aunque queda por encima de la media (5.5), el aspecto del \textbf{diseño gráfico} no ha sido muy valorado por los usuarios en su evaluación, lo cual quiere decir que es necesario modificar las pantallas para hacerlas más atractivas de cara al usuario y con mejor diseño. Para esto se utilizarán técnicas de diseño de interfaces de usuario mas avanzadas.\\\\\\

Los aspectos de  \textbf{facilidad de us} o y  \textbf{aprendizaje} han obtenido una notable valoración por parte de los usuarios, verificando de este modo uno de los principios que desde el inicio de la aplicación se ha intentado seguir; que sea fácil de usar y aún más fácil de recordar, por lo que en todo momento se ha intentado minimizar los pasos a realizar para ejecutar una tarea pretendiendo que sean acciones simples y con poco texto para de este modo evitar confundir al usuario al recordar como se realizaba una determinada tarea.\\

El aspecto más valorado por los usuarios para la aplicación ha sido la \textbf{satisfacción}, obteniendo una media de 7.25, lo que nos dice que en general los usuarios se sienten satisfechos con las diferentes funcionalidades de la aplicación, ya que cumple con los objetivos deseados y planificados anteriormente. Pero cabe destacar que aunque la aplicación haya obtenido una buena nota de medida en la mayoría de los aspectos, esta evaluación nos permitirá poder mejorar la aplicación en los puntos en los que más énfasis se ha echo, y de este modo poder seguir realizando el refinamiento del sistema con sucesivas evaluaciones y cumpliendo en todo momento con el diseño centrado en el usuario que se ha llevado desde el principio.

		
\end{document}
